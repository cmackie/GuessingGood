\documentclass{article}
% if you need to pass options to natbib, use, e.g.:
% \PassOptionsToPackage{numbers, compress}{natbib}
% before loading nips_2017
%
% to avoid loading the natbib package, add option nonatbib:
% \usepackage[nonatbib]{nips_2017}

\usepackage[final]{nips_2017}

% to compile a camera-ready version, add the [final] option, e.g.:
% \usepackage[final]{nips_2017}

\usepackage[utf8]{inputenc} % allow utf-8 input
\usepackage[T1]{fontenc}    % use 8-bit T1 fonts
\usepackage{hyperref}       % hyperlinks
\usepackage{url}            % simple URL typesetting
\usepackage{booktabs}       % professional-quality tables
\usepackage{amsfonts}       % blackboard math symbols
\usepackage{nicefrac}       % compact symbols for 1/2, etc.
\usepackage{microtype}      % microtypography

\title{Guessing Good: Applying Machine Learning to Boolean Satisfiability}

% The \author macro works with any number of authors. There are two
% commands used to separate the names and addresses of multiple
% authors: \And and \AND.
%
% Using \And between authors leaves it to LaTeX to determine where to
% break the lines. Using \AND forces a line break at that point. So,
% if LaTeX puts 3 of 4 authors names on the first line, and the last
% on the second line, try using \AND instead of \And before the third
% author name.

\author{
    Nathaniel Yazdani \quad Christopher A. Mackie\\
    Paul G. Allen School of Computer Science and Engineering\\
    University of Washington\\
    Seattle, WA 98105\\
    \texttt{\{nyazdani, mackic\}@cs.washington.edu}\\
}

\begin{document}
% \nipsfinalcopy is no longer used

\maketitle

%\section*{Project Proposal}

%We will be using data exclusively from 3-clause 3-SAT.
%Restricting the problem to 3-clause 3-SAT formulae avoids distracting problems about the sparsity of collected data and
%enables us to more conclusively answer the research question: \emph{Can a machine-learnt probabilistic model be an
%effective augmentation to a modern SAT solver}? We will prune the set of problems in this domain based on their
%modularity, $Q$, a measure of community structure.
%In particular, we will include only problems with $Q > 0.13$, as these are more representative of realistic SAT
%problems [1].
%
%The form of this data for a particular problem will be the problem itself, a log of the semantic actions (\textit{e.g.},
%case splitting, unit propagation) taken by an instrumented version of MiniSAT, and a set of variable assignments which
%satisfy the problem (or the empty set if it is unsatisfiable).
%This data is included with this proposal.

\subsection*{Dataset}

We will be using benchmark SAT problems from the international SAT
Competition\footnote{\url{http://www.satcompetition.org/}}. These problems
consist of SAT problems of varying degrees of difficulty, and their
satisfiability. We will only use problems from the \textbf{industrial} and
\textbf{application} categories, as randomly generated and crafted SAT instances
are well known to be unrepresentative of real-world SAT problems [1].

\subsection*{Project Idea}

Our project aims to answer the following research question: \emph{Can a modern
SAT solver leverage a probabilistic model to improve the accuracy of its random
guesses when solving real-world SAT instances?} Real-world SAT instances have
significant internal structure [1] exploited by existing techniques for variable
prioritization, but no such techniques exist for guessing the variable
assignments.

We will proceed in two phases. In the first phase, we will train a model to
produce a vector of variable assignment probabilities (where element $i$ is the
probability that the $i$\textsuperscript{th} variable is true) for an input SAT
formula. On its own, we expect this model to perform poorly as a predictor of
satisfying assignments, but the engineering and experience will directly tie
into the second phase.

In the second phase, we will integrate the model into a modern SAT solver, to
use the model's predictions only when forced to make a random guess for a
particular variable assignment. In this phase, we will also design a conjugate
distribution so that the solver may update the model with facts logically
deduced from earlier decisions (\textit{e.g.}, unit propagation and conflict
analysis).

%The first will be trained on our dataset to produce an initial prior for the problem space.
%The second will continually update this prior/posterior as an instrumented version of MiniSAT solves the problem.
%This model will be used to inform MiniSAT which variable it should make a guess for next (basically inferring a decision
%tree for minimal backtracking).
%When MiniSAT does backtrack, we can update the model with the conflict clause, which is typically discarded due to
%memory constraints.

\subsection*{Software}

We will instrument MiniSAT\footnote{\url{https://github.com/niklasso/minisat}},
a modern SAT solver, to implement our guided search.
We chose MiniSAT for its
thorough documentation and past performance in the SAT Competition.

\subsection*{Papers to Read}

Our selected papers are listed in References. [2-4] are most relevant for
the execution of this project, whereas [1] provides background knowledge on the
structure found in real-world SAT instances.

\subsection*{Milestone Plan}

We plan to complete the first phase by the milestone. %We hope to have extracted
%several useful features to use in our more sophisticated technique from this
%experiment.

\section*{References}
\label{references}

\small

[1] Audemard, G., Fischmeister, S., Ganesh, V., Newsham, Z., \& Simon, L. (2014).
Impact of Community Structure on SAT Solver Performance. \textit{SAT}.

[2] Blaschko, M.B., \& Flint, A. (2012).
Perceptron Learning of SAT. \textit{NIPS}.

[3] Czarnecki, K., Ganesh, V., Liang, J.H., \& Poupart, P. (2016).
Learning Rate Based Branching Heuristic for SAT Solvers. \textit{SAT}.

[4] Ganesh, V., Near, J.P., Rinard, M., \& Singh, R. (2009).
AvatarSAT: An Auto-Tuning Boolean SAT Solver. \textit{MIT Technical Report}.

\end{document}
